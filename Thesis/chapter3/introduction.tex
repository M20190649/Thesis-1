\section{Introduction}
Information networks such as social networks, biological networks and
phone-call networks are typically modeled as graphs \cite{chen2008graph}
where the vertices correspond to objects and the edges
capture the relationships between these objects.
For instance, in social networks, every user is represented by
a vertex and the friendship between two users is reflected by an edge between
the vertices. In addition, a user's profile can be maintained as
the vertex's attributes. Such graphs contain a wealth of valuable 
information which can be analyzed to discover interesting patterns. 
For example, we can find the top-k influential users who can 
reach the most number of friends within 2 hops. With increasingly
larger network sizes, it is becoming significantly challenging to 
query, analyze and process these graph data. Therefore, there is an urgent need 
to develop effective and efficient mechanisms over graph data to draw out
information from such data resources.
 
Traditionally, in relational DBMS, window functions have been commonly
used for data analytics \cite{cao2012optimization, bellamkonda2013adaptive}. Instead of performing analysis (e.g. ranking, 
aggregate) over the entire data set, a window function returns for each 
input tuple a value derived from applying the function over a window of 
neighboring tuples. For instance, users may be interested in finding 
each employee's salary ranking within the department. Here,
each tuple's neighbors are essentially records from the same department.

%Interestingly, we find that the locality feature in graph scenario has even more importance than in relational data. In graph data, the graph structure indicates the relationship among different objects. In graph analysis, it is often useful to quantify a structural range to each vertex and perform analytic over the range. For instance, in a social network, it is important to detect one person's social position and influence in his/her social community. In this paper, we are motivated to extend the window queries in traditional SQL into the graph analysis. However, the window definition in the SQL is no longer applicable in graph context, as it does not capture the graph structure information which has a strong meaning in the graph data. 

Interestingly, the notion of window functions turns out to be not uncommon
in graph data. For instance, in a social network, it is important to detect 
a person's social position and influence among his/her social community. 
The ``social community'' of the person is essentially his/her ``window'' 
comprising neighbors derived from his/her k-hop friends.
However, as illustrated in this example, the structure of a graph
plays a critical role in determining the neighboring data of a vertex.
In fact, it is often useful to quantify a structural range to each vertex 
and then perform analytics over the range. 
Surprisingly, though such a concept of window functions has been widely
used, the notion has not been explicitly formulated. 
In this paper, we are motivated to extend the window queries in traditional 
SQL for supporting graph analysis. However, the window definition in 
SQL is no longer applicable in a graph context, as it does not capture 
the graph structure information.
% which has a strong meaning in the graph data. 
Thus, we seek to formulate the notion of graph windows and to develop
efficient algorithms to process them over large scaled graph structures. 

We have identifed two instantiations of graph windows, namely 
$k${\em -hop} and {\em topological} windows. 
We first demonstrate these window semantics with the following examples. 
\begin{example}
\label{query:linkedin-2-hop-window}
\emph{(k-hop window)} In a social network ( such as Linked-In and Facebook etc.), users are normally modeled as vertices and connectivity relationships are modeled as edges. In social network scenario, it is of great interest to summarize the most relevant connections to each user such as the neighbors within 2-hops. Some analytic queries such as summarizing the related connections' distribution among different companies, and computing age distribution of the related friends can be useful. In order to answer these queries, collecting data from every user's neighborhoods within 2-hop is necessary.
\end{example}

\begin{example}
\label{query:bio-dag-window}
\emph{(Topological window)} In biological networks ( such as Argocyc, Ecocyc etc.\cite{keseler2005ecocyc}), genes, enzymes and proteins are vertices and their dependency in a pathway are edges. Because these networks are directed 
and acyclic, in order to study the protein regulating process, one may be 
interested to find out the statistics of molecules in each protein 
production pathway. For each protein, we can traverse the graph to find 
every other molecule that is in the upstream of its pathway. Then we can group and count the number of genes and enzymes among those molecules. 
\end{example}

A common feature among these examples is that data aggregation is needed based on a set of vertices (which is the {\em graph window}) 
defined according to each vertex.  To illustrate, in example \ref{query:linkedin-2-hop-window}, every user needs to gather data from its friends and friends-of-friends. 
The \emph{2-hop neighbors} form its window. Likewise, in example \ref{query:bio-dag-window}, every protein needs to count the number of particular type of genes preceding it in the regulating pathway. For every protein, the set of
\emph{preceding molecules} forms its window. 

To support the analyses in the above-mentioned examples, we propose a new 
type of query, \emph{Graph Window Query} (GWQ in short),
over a data graph. \emph{GWQ} is defined with respect to a graph structure 
and is important in a graph context. Unlike the traditional window in SQL, 
we identify two types of useful graph windows according to the 
graph structures, namely k-hop Window $W_{kh}$ and Topological Window $W_t$. 
A k-hop window forms a window for one vertex by using its k-hop neighbors. 
k-hop neighbors are important to one vertex, as these are the vertices 
showing structural closeness as in Example 1. The k-hop neighbors window 
we define here is similar to the egocentric-network in network analysis \cite{burt2009structural} \cite{mondal2014eagr}. A topological window, on the
other hand, forms a window for one vertex by using all its preceding 
vertices in a directed acyclic graph. The preceding vertices of one vertex are normally those which influence the vertex in a network as illustrated in Example 2. 

To the best of our knowledge, existing graph databases or graph query languages do not directly support our proposed GWQ. There are two major challenges in processing GWQ. First, we need an efficient scheme to  calculate the window of each vertex. Second, we need
efficient solutions to process the aggregation over a large number 
of windows that may overlap. This offers opportunities to share the 
computation. However, it is non-trivial to address these two challenges.  

For $k$-hop window like query, the state-of-the-art processing algorithm
is EAGR \cite{mondal2014eagr}. EAGR builds an overlay graph including 
the shared components of different windows. This is done 
in multiple iterations, each of which performs the following.
First, EAGR sorts all the vertices according to their $k$-hop 
neighbors based on their lexicographic order. 
Second, the sorted vertices are split into equal sized chunks each of which is further built as one frequent-pattern tree to mine the shared components. 
However, EAGR requires all the vertices' $k$-hop 
neighbors to be pre-computed and resided in memory during 
each sorting and mining operation;
otherwise, EAGR incurs high computation overhead if the pre-computed structure needs to be shuffled to/from disk.
This limits the efficiency and scalability of EAGR.
 For instance, 
a LiveJournal social network graph\footnote{Available at http://snap.stanford.edu/data/index.html, which is used in \cite{mondal2014eagr}} 
(4.8M vertices, 69M edges) generates over 100GB neighborhood information 
for k=2 in adjacency list representation. In addition, the overlay graph construction is not a one-time task,
but periodically performed after a certain number of structural updates in order to maintain the overlay quality. The high memory consumption renders the scheme impractical 
when $k$ and the graph size increases.

In this paper,
we propose \textit{Dense Block Index (DBIndex)} 
to process queries efficiently.
Like EAGR, DBIndex seeks to exploit common
components among different windows to salvage 
partial work done. However, unlike EAGR,
we identify the window similarity utilizing a hash-based 
clustering technique instead of sorting. This ensures 
efficient memory usage, as the window information of each vertex can 
be computed on-the-fly. On the basis of the clusters, we develop different optimizations 
to extract the shared components which result in an efficient index construction. 

Moreover, we provide another \emph{Inheritance Index (I-Index)} tailored 
to topological window query. I-Index differentiates itself from
DBIndex by integrating more descendant-ancestor relationships 
to reduce repetitive computations. This results in
%to reduce the repetitive computation. This results in
more efficient index construction and query processing.  

 
Our contributions are summarized as follows:
\begin{itemize}
\item{We propose a new type of graph analytic query, \emph{Graph Window Query} and formally define two graph windows: k-hop window and topological window. We illustrate how these window queries would help users better query and
understand the graphs under these different semantics.}

\item{ To support efficient query processing, we further propose two different types of indices: \emph{Dense Block Index} (DBIndex) and \emph{Inheritance Index} (I-Index). The
\emph{DBIndex} and \emph{I-Index} are specially 
optimized to support k-hop window and topological window query processing. 
We develop the indices by integrating the window aggregation sharing techniques to salvage partial work done for efficient computation. In addition, we develop space and performance efficient techniques for index construction. 
Compared to EAGR \cite{mondal2014eagr}, the state-of-the-art index method for k-hop window queries, our DBIndex is much more memory efficient and scalable towards handling the large-scale graphs. }

\item{We perform extensive experiments over both real and synthetic datasets
with hundreds of millions of vertices and edges on a single machine. Our experiments 
indicate that our proposed index-based algorithms outperform the naive non-index
algorithm 
by up to four orders of magnitude. In addition, our experiments also show 
that DBIndex is superior over EAGR in terms of both
scalability and efficiency. In particular, 
DBIndx saves up to 80\% of indexing time as 
compared to EAGR, 
and performs well even when EAGR fails due
to memory limitations. 
}
\end{itemize}
