\section{Summary}\label{gw:sec:concl}
In this chapter, we studied the neighborhood analytics in attributed graphs.
We leverage the distance and comparison neighborhoods to propose
two graph windows: $k$-hop window and topological window.
Based on the two window definitions, we proposed a new type of graph analytic query, \emph{Graph Window Query} (GWQ).
%
%In this chapter, we proposed a new type of graph analytic query, \emph{Graph Window Query}. We formally defined two instantiations of graph windows: $k$-hop window and topological window. 
Then, we studied GWQ processing for large-scale graphs. In particular,
we developed the Dense Block Index (DBIndex) to facilitate efficient processing of both types of graph windows. Moreover, we proposed the Inheritance Index (I-Index) that exploits a containment property of DAG to enhance the query performance of topological window queries. 
%The two indexes integrate window aggregation sharing techniques to salvage partial work done, which is both space and query efficient. 
Last, we conducted extensive experimental evaluations over both large-scale real and synthetic datasets. The experimental results showed the efficiency and scalability of our proposed indexes. 


%There remain many interesting research problems for graph window analytics. 
%As part of our future work, we plan to explore structure-based window aggregations
%which are complex than attribute-based window aggregations.
%In structure-based aggregations, $W(v)$ refers to a subgraph of $G$ instead of a set of vertexes,
%and the aggregation function $\Sigma$ (e.g., centrality, PageRank, and {graph aggregation} \cite{wang2014pagrol,zhao2011graph}) operates on the structure of the subgraph $W(v)$. 
%Another interesting direction that we plan to investigate is the support of efficient structure updates (e.g., edge deletions and insertions). 