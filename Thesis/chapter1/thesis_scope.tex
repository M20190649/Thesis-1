\section{Scope of the Thesis}
In this thesis, we explore the neighborhood analytics in
three prevalent data domains, namely \textbf{attributed graph},
\textbf{sequence data} and \textbf{trajectory}. 
To provide useful analytics, we define the following
two intuitive instances of the neighborhood function:
%Since the neighborhood analytics could be very broad,
%this thesis only focus on the following 
%two intuitive neighborhood functions:

\textbf{Distance Neighborhood}: the neighborhood is defined based on numeric distance, that is $\mathcal{N}(o_i,K) = \{o_j | \mathtt{dist}(o_i,o_j) \leq K \}$, where $\mathtt{dist}$ is a distance function and $K$ is a distance threshold.

\textbf{Comparison Neighborhood}: the neighborhood is defined based on the comparison of objects, that is $\mathcal{N}(o) = \{o_i | o.a_m \ \mathtt{cmp} \ o_i.a_m\}$, where $a_m$ is an attribute of object
and $\mathtt{cmp}$ is a binary comparator (i.e., $=,<,>,\leq,\neq,\geq$).

Despite the simpleness of the these two neighborhood functions, they can weave many useful analytics as we shall see in the remaining part of the thesis.
%
%There could be other types of neighborhood functions with different constraints. 
%However, despite the simpleness of these two neighborhood functions, they 
%are indeed versatile in representing many useful analytics.

%In particular, we looked at three most prevalent data domains, namely \textbf{attributed graph},
%\textbf{time series} and \textbf{trajectory}. 
%We then categorize two intuitive neighborhood functions as follows:
%
%\textbf{Distance Neighborhood}: the neighborhood is defined based on numeric distance, that is $\mathcal{N}(o_i,K) = \{o_j | \mathtt{dist}(o_i,o_j) \leq K \}$, where $\mathtt{dist}$ is a distance function and $K$ is a distance threshold.
%
%\textbf{Comparison Neighborhood}: the neighborhood is defined based on the comparison of objects, that is $\mathcal{N}(o) = \{o_i | o.a_m \ \mathtt{cmp} \ o_i.a_m\}$, where $a_m$ is an attribute of object
%and $\mathtt{cmp}$ is a binary comparator.
%
%There could be other types of neighborhoods with more fine-grained or more general definitions. 
%In this thesis, we demonstrate
%that these two simple neighborhood definitions
%joint with traditional analytic functions are already versatile 
%in various domains. 
%\revised{They are cable to both express existing queries and devise novel emerging 
%queries that are practically useful.
%}


