\section{Summary} \label{sec:conclusion}
In this chapter, we looked at the neighborhood analytics in
sequence data. We leverage the joint distance and comparison neighborhood 
functions to design the novel \emph{ranked-streak} which quantifies the strikingness of
a streak. We then formulated the \emph{$k$-Sketch} query which aims to best summarize a subject's history using $k$ ranked-streaks.
We studied the $k$-Sketch query processing in both offline and online scenarios,
and propose efficient solutions to cope each scenario. 
In particular, we designed novel streak-level pruning techniques and a $(1-1/e)$-approximate algorithm to achieve efficient processing in offline. Moreover, we designed a $1/8$-approximate algorithm for the online sketch maintenance.
Our comprehensive experiments demonstrated the efficiency of our solutions and a human study confirmed the effectiveness of the $k$-Sketch query.

%In this chapter, we look at the problem of automatically summarize 
%a subject's history using newsworthy stories.
%We group consecutive events into streaks and propose a novel idea of \emph{ranked-streak} to represent the strikingness.
%We then formulate the \emph{$k$-Sketch} query which aims to best summarize a subject's history using $k$ ranked-streaks.
%We study the $k$-Sketch query processing in both offline and online scenarios,
%and propose efficient solutions to cope each scenario. In particular, we design novel streak-level pruning techniques and a $(1-1/e)$-approximate algorithm to achieve efficient processing in offline. Then we design a $1/8$-approximate algorithm for the online sketch maintenance.
%Our comprehensive experiments demonstrate the efficiency of our solutions and a human study confirms the effectiveness of the $k$-Sketch query.

%Our work opens a wide area of research in computational journalism. 
%In our next step, we aim to extend the event sources from sensor data 
%to non-schema data such as tweets in social networks. We also aim to 
%investigate subject association models to automatically generate 
%news themes involving multiple subjects.
%We further wish to explore leveraging big data technology to support fast-growing 
%event data in journalism. 